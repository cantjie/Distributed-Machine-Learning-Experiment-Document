\chapter{环境配置}

本章将主要介绍分别使用本地计算资源、深研院计算资源和华为云计算资源时构建环境的方法。

\section{使用本地环境与GPU}

使用本地计算资源可以不收网络链接状况约束,随时随地调试程序,对于简单的项目,本地调试也可能更省时间。

本节以助教所使用的计算机为例,展示环境配置过程。助教使用的计算机系统与配置为:
\begin{itemize}
    \item 系统:windows10专业教育版;22H2
    \item 处理器:Intel(R) Core(TM) i7-8700 CPU 
    \item 内存:16GB
    \item 显卡:NVIDIA GeForce RTX 2060
    \item 编辑器:Visual Studio Code
\end{itemize}

\subsection{安装CUDA工具箱}

对于包含英伟达显卡的计算机,我们推荐首先安装CUDA工具包以使用GPU加速计算。

\textcolor{red}{\emph{注意,仅包含英伟达GPU的计算机需要安装CUDA工具箱以使用GPU加速计算。使用核显或AMD显卡的计算机再后续步骤中使用CPU计算即可}}

GPU型号、CUDA工具包、PyTorch版本相互关联。因此需要一起规划好。

\url(https://developer.nvidia.com/zh-cn/cuda-gpus)
查看得到我的显卡的2060的算力为7.5。

\begin{figure}[htbp]
    \centering
    \includegraphics[width=0.7\textwidth]{figures/nvidia-rtx-2060-capability.png}
    \caption{CAPTION holder}
    \label{LABEL holder}
\end{figure}


\url{https://en.wikipedia.org/wiki/CUDA}
查看得到支持我显卡的CUDA版本为$\geq$10.0
\begin{figure}[htbp]
    \centering
    \includegraphics[width=0.7\textwidth]{figures/corresponding-cuda-version.png}
    \caption{CAPTION holder}
    \label{LABEL holder}
\end{figure}

\url{https://docs.nvidia.com/cuda/cuda-toolkit-release-notes/index.html}
同时CUDA对显卡驱动的最低版本也提出了要求,但显卡驱动对CUDA向下兼容,因此一般安装了最近发布的显卡驱动版本即可,无需与CUDA版本特别对应。


\url{https://pytorch.org/get-started/locally/}
最后要注意,PyTorch并不一定支持最新的CUDA版本,因此安装前再去PyTorch上看一眼PyTorch支持哪些CUDA版本。
\begin{figure}[htbp]
    \centering
    \includegraphics[width=0.7\textwidth]{figures/cuda-version-constrained-by-pytorch.png}
    \caption{CAPTION holder}
    \label{LABEL holder}
\end{figure}

我们发现PyTorch最高支持到CUDA 11.7,满足显卡算力对CUDA版本$\geq$10.0的要求,因此我们可以选择安装CUDA 11.7。

选择对应版本CUDA安装包并下载安装, 安装过程略。
\url{https://developer.nvidia.com/cuda-toolkit-archive}

\begin{figure}[htbp]
	\centering
	\includegraphics[width=0.7\textwidth]{figures/select-cuda-version.png}
	\caption{CAPTION holder}
	\label{LABEL holder}
\end{figure}

在这一步完成后,我们打开终端输入\graylstinline{nvcc -V}以及\graylstinline{nvidia-smi}应当分别能看到图\ref{fig:nvcc-v-install-success}和图\ref{fig:nvidia-smi-install-success}类似的输出,这说明我们安装完成。

\begin{figure}[htbp]
	\centering
	\includegraphics[width=0.7\textwidth]{figures/nvcc-v-install-success.png}
	\caption{caption:nvcc-v-install-success}
	\label{fig:nvcc-v-install-success}
\end{figure}

\begin{figure}[htbp]
	\centering
	\includegraphics[width=0.7\textwidth]{figures/nvidia-smi-install-success.png}
	\caption{caption:nvidia-smi-install-success}
	\label{fig:nvidia-smi-install-success}
\end{figure}

\subsection{安装Anaconda}

我们可能同时有多个项目或作业在处理,而不同的项目或作业可能使用了不同python版本、不同的工具包等,为了避免冲突,我们通常会为每一个项目或作业指定一个虚拟环境,以使得各个环境之间互不干扰。为此,我们Anaconda以创建并管理虚拟环境。

安装过程参考官网文档即可:
\url{https://docs.anaconda.com/anaconda/install/windows/}

安装完成后启动终端,输入\graylstinline{conda -V},如正确显示conda版本则说明安装成功。
\begin{figure}[htbp]
	\centering
	\includegraphics[width=0.7\textwidth]{figures/conda-install-success.png}
	\caption{CAPTION holder}
	\label{LABEL holder}
\end{figure}

\subsection{创建虚拟环境并安装PyTorch}

安装完成conda后,我们新建一个预装了Python的、用来完成本门课程的虚拟环境。

需要注意的是,PyTorch和Python版本也需要对应,在\url{https://github.com/pytorch/vision#installation}中,我们发现torch 1.13要求python 介于3.7.2和3.10之间。

\begin{figure}[htbp]
	\centering
	\includegraphics[width=0.7\textwidth]{figures/pytorch-python-version-compability.png}
	\caption{CAPTION holder}
	\label{LABEL holder}
\end{figure}

打开终端,输入下面命令以利用conda新建环境,
\begin{lstlisting}
    $ conda create --name <envname> python=3.9
    $ 
\end{lstlisting}
将其中<envname>改成自定义的环境名称,如助教自己选择的distributedml。

新建完成后,通过\graylstinline{conda activate <envname>}进入环境。在pytorch官网安装页面\url{https://pytorch.org/get-started/locally/}选择对应的pytorch版本、系统版本等,复制给出的命令并运行。

\begin{figure}[htbp]
	\centering
	\includegraphics[width=0.7\textwidth]{figures/pytorch-install-command.png}
	\caption{CAPTION holder}
	\label{fig:pytorch-install-command}
\end{figure}

安装完成后,进入Python就可以\graylstinline{import torch}了,如图\ref{fig:pytorch-install-success}.

\begin{figure}[htbp]
	\centering
	\includegraphics[width=0.7\textwidth]{figures/pytorch-install-success.png}
	\caption{caption:pytorch-install-success}
	\label{fig:pytorch-install-success}
\end{figure}


% \subsection{使用VSCode编辑python文件}

% 以实验一为例,VSCode安装Python插件后,


\section{使用虚拟环境与本地GPU}

上面的本地环境配置不可为不复杂,CUDA、显卡型号、显卡驱动、PyTorch、Python等版本需要手动一一对应起来安装。那有没有什么更简单的利用本机GPU计算资源的方法呢?

在这一节,我们介绍直接利用Docker镜像搭配环境的方法。

\subsection{安装并配置Docker引擎}

首先在官网下载安装包\url{https://docs.docker.com/desktop/install/windows-install/},安装过程略。

在安装完成后启动Docker Desktop,在windows下,很可能会报错(具体内容是啥助教忘了截图了),一般错误的原因是缺少wsl2和hyper-v。

为了启用hyper-v,在控制面板中按照图\ref{fig:turn-on-hyper-v}中的操作选中Hyper-V并确定。
\begin{figure}[htbp]
	\centering
	\includegraphics[width=0.7\textwidth]{figures/turn-on-hyper-v.png}
	\caption{caption:turn-on-hyper-v}
	\label{fig:turn-on-hyper-v}
\end{figure}

为了启用wsl2,参考\url{https://learn.microsoft.com/en-us/windows/wsl/install},在终端下输入\graylstinline{wsl --install}等待安装完成即可。

\begin{figure}[htbp]
	\centering
	\includegraphics[width=0.7\textwidth]{figures/docker-mirrors-setting.png}
	\caption{caption:docker-mirrors-setting}
	\label{fig:docker-mirrors-setting}
\end{figure}

安装完成后启动Docker Desktop,为了加速下载,可以按照图\ref{fig:docker-mirrors-setting}所示方法为Docker指定国内镜像服务器,即在原本的配置中加入如下内容。
\begin{lstlisting}
    "registry-mirrors": [
        "http://hub-mirror.c.163.com",
        "https://docker.mirrors.ustc.edu.cn",
        "https://registry.docker-cn.com"
    ]
\end{lstlisting}

启动终端,输入\graylstinline{docker --version},如图\ref{fig:docker-install-success},正常返回Docker版本就说明安装成功了。
\begin{figure}[htbp]
	\centering
	\includegraphics[width=0.7\textwidth]{figures/docker-install-success.png}
	\caption{caption:docker-install-success}
	\label{fig:docker-install-success}
\end{figure}


\subsection{搜索并下载PyTorch镜像}

Dockerhub是一个共享镜像的平台\url{https://hub.docker.com/}。所谓镜像,类似于一个操作系统的iso文件:我们拿到iso文件后可以创建使用该操作系统的虚拟机;而当我们拿到镜像后,也可以利用该镜像创造一个使用该镜像的容器,即容器是一个镜像的实例。

因此,如果有人在某个容器中把CUDA、PyTorch、Python等环境都配置好,并打包成镜像共享给我们,我们就可以免去复杂的安装过程,从而直接使用镜像生成容器,在容器中直接运行我们所写的脚本。

在DockerHub中,我们搜索\graylstinline{pytorch/pytorch},可以找到对应的这个镜像\url{https://hub.docker.com/r/pytorch/pytorch}。点击网页中的Tags标签页,我们可以从图\ref{fig:pytorch-image-tags-web}看到这个镜像就是已经把PyTorch和CUDA安装好了的,我们直接使用这个镜像就好啦!

\begin{figure}[htbp]
	\centering
	\includegraphics[width=0.7\textwidth]{figures/pytorch-image-tags-web.png}
	\caption{caption:pytorch-image-tags-web}
	\label{fig:pytorch-image-tags-web}
\end{figure}

下载这个镜像前,还需要登录的。首先去注册个账号,然后打开终端,输入\graylstinline{docker login}登录。

然后就可以通过这条命令下载这个镜像了:

\begin{lstlisting}
    $ docker pull pytorch/pytorch:1.13.1-cuda11.6-cudnn8-runtime
    $
\end{lstlisting}


这个镜像比较大,下载需要一点时间。完成后,我们再输入\graylstinline{docker image list}就可以看到这个镜像了,见图\ref{fig:docker-image-list-pytorch}。
\begin{figure}[htbp]
	\centering
	\includegraphics[width=0.7\textwidth]{figures/docker-image-list-pytorch.png}
	\caption{caption:docker-image-list-pytorch}
	\label{fig:docker-image-list-pytorch}
\end{figure}

\subsection{启动容器}

下载完镜像,我们该通过这个镜像启动一个容器了,我们需要到容器里看看这个容器里面是不是有我们需要的环境。

打开终端,输入
\begin{lstlisting}
    $ docker run -it pytorch/pytorch:1.13.1-cuda11.6-cudnn8-runtime
\end{lstlisting}
我们发现我们进入了一个linux系统,进去运行一下\graylstinline{nvidia-smi}试试,诶,怎么command not found,看不到显卡。这是因为容器启动时没有给他指定GPU。我们输入\graylstinline{exit},然后加上GPU参数再试一下
\begin{lstlisting}
    $ docker run --gpus all -it pytorch/pytorch:1.13.1-cuda11.6-cudnn8-runtime 
\end{lstlisting}

进入容器后,我们输入\graylstinline{nvidia-smi}等命令,查看运行结果,如图\ref{fig:docker-pytorch-container-env-check}所示,发现正是我们所需要的环境。
\begin{figure}[htbp]
	\centering
	\includegraphics[width=0.7\textwidth]{figures/docker-pytorch-container-env-check.png}
	\caption{caption:docker-pytorch-container-env-check}
	\label{fig:docker-pytorch-container-env-check}
\end{figure}

可是如何使用这个环境呢,我们留到完成具体实验内容的时候再来讲。


\subsection{限制Docker内存占用(TODO,可选)}

TODO


\section{华为云计算资源}




\begin{lstlisting}[language=Python, caption=Python example]
    some python code here?
            return model
\end{lstlisting}